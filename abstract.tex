


\thispagestyle{empty}
\pagestyle{empty} 

\begin{abstract}

\hspace{0.9cm} When a customer of say a corporate telecom major like Airtel decreases spending on his or her number, they tend to provide more offers to the customer. Now this logic may sound straightway outrageous but this is true based on churn models. Churning refers to customers leaving a service and switching to another provider. The standard industry practice is to generate churn models for every consumer, so that they can provide business intelligence to increase customer retention and spending. Thus this translates to greater profit for a company.
Although such processes can be carried out with a GPU, the software that exists like SAP handles everything on the CPU. This leads to longer delivery period for business intelligence and analytics which in turn means less time spent doing business and more time spent waiting. GPUs can generate huge number of threads and thus introduce parallelism in this area. In addition to the above advantage the churn models have assisted learning and cannot learn on its own unlike Deep Neural Networks which apply the concept of machine learning. This project aims to create a Windows based general purpose application to try predicting churn with better analytics.
The scope of the project is limited to test data sets freely available on the internet as real data can only be obtained from corporates who are rather unlikely to share sensitive data with non-employees. Thus we aim to create a churn model that uses GPU based DNNs which not only will improve execution time for large data sets, it will also prove to be a huge plus point for corporate customers who want to use this software.

\end{abstract} 

