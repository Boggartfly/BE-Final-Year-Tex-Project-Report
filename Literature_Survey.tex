

\chapter{LITERATURE SURVEY}

\section{Literature Survey}

\hspace{0.9cm} The highlight of the paper is the idea of using the normalized
data as an image and using the generated image as input
to the neural network. This concept was inspired by the paper
on churn data model viewed as periods of time \cite{citation-2}.
\subsection{Dataset Used for the Experiment}
Our experiment required data to be processed and used
as an input. This was obtained online from bigml \cite{citation-12}. We,
thus, obtained a labelled dataset, which enables us to use
supervised learning approach for our neural network. Here,
our sample dataset appears to be monthly rather than weekly
as compared to the dataset used by Wangperpong et al. \cite{citation-4}.
Our sample dataset consists of rows - customers and columns
as attributes associated with customers. The attributes, which
are significant, are not identified in our dataset, but we only
consider numeric values while trying to predict the churn rate.
\subsection{Pattern Matching and Clustering Neural Networks Using Supervised Learning}
Furthermore, there has been a lot of work done in pattern
matching via supervised learning. Schwenker et al. \cite{citation-3} describe
this concept very clearly in their detailed analysis of the
topic. This approach to supervised learning is heavily applied
to neural networks. Neural networks thus find far and
wide reaching applications that can have major impacts on
society like the traffic camera system \cite{citation-5}.

\section{Problem Definition}

\hspace{0.9cm} Customer retention is a huge requirement for any service based business. This is especially more emphasised in the Telecommunications industry where cellular carriers must retain customers to remain profitable or risk losing money. Churn analysis aims to predict the churn which is the act of a consumer leaving or switching service. Thus these predictions offer a chance to act on potential loss of subscribers. But this deep analysis of data using Neural Networks requires immense computing power and time both of which are critical in a subscription model of business. The problem is to create a technique to demonstrate speedup in the analysis process of churn data and then provide predictions with accuracy over 90\% reliably.

